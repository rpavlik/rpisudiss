% !TEX TS-program = pdflatex
% !TEX encoding = UTF-8 Unicode

\documentclass[draft]{rpisudiss}
\usepackage[utf8]{inputenc} % set input encoding (not needed with XeLaTeX)
\usepackage[T1]{fontenc}

\usepackage{graphicx} % support the \includegraphics command and options


\usepackage{url}
\usepackage{prettyref}


%%% END Article customizations

%%% The "real" document content comes below...


\title{This is the title}


\author{Ryan Andrew Pavlik}
%\date{} % Activate to display a given date or no date (if empty),
         % otherwise the current date is printed 

\begin{document}
\thispagestyle{empty}
\maketitle

\pagestyle{fancy}
\tableofcontents{}

\phantomsection
\addcontentsline{toc}{chapter}{List of Figures}
\listoffigures

\phantomsection
\addcontentsline{toc}{chapter}{List of Tables}
\listoftables

%\begin{abstract}
%This is an abstract. How abstract!
%\end{abstract}

\chapter{Intro}


A wide variety of software frameworks for building virtual reality
applications have been developed. The CAVE~Library initially developed
for use with the CAVE~Automated~Virtual~Environment \cite{Cruz-Neira1993}
is an example of early work in the systems category of virtual reality
frameworks. It has evolved into a commercial solution integrating
clustering support and focusing on multi-screen application development.
VR~Juggler introduced a highly modular architecture for VR applications
to provide a ``virtual platform'' for development and execution
on diverse systems \cite{Bierbaum2001,Bierbaum2005}.

frameworks. It has evolved into a commercial solution integrating
clustering support and focusing on multi-screen application development.
VR~Juggler introduced a highly modular architecture for VR applications

\section{Bla bla}
Later development
extended its use from high-end graphics systems to commodity computer
clusters \cite{Allard2002,Bierbaum2005}.


frameworks. It has evolved into a commercial solution integrating
clustering support and focusing on multi-screen application development.
VR~Juggler introduced a highly modular architecture for VR applications

\subsection{bla bla bla}
The commercial
VR authoring environment Virtools%
\footnote{\url{http://www.virtools.com/}%
} integrates a custom scripting language, VSL, for content creation.

frameworks. It has evolved into a commercial solution integrating
clustering support and focusing on multi-screen application development.
VR~Juggler introduced a highly modular architecture for VR applications
\chapter{middle}

A VR~JuggLua application uses both the osgLua module
and the VR~Juggler bindings included in the VR~JuggLua framework
to access a complete set of virtual reality functionality from Lua
(Figure \ref{fig:System-diagram}). 
\begin{figure}
\begin{centering}
\includegraphics[width=1\columnwidth]{accessdiagram}
\par\end{centering}

\caption{\label{fig:System-diagram}System diagram}
\end{figure}


\chapter{Conclusion}

aefawefwaefadfwdcw


\bibliographystyle{abbrv}
\bibliography{vrjugglua}

\end{document}
